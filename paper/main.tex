\documentclass{article}
\usepackage{arxiv}

\usepackage[utf8]{inputenc}
\usepackage[english, russian]{babel}
\usepackage[T1]{fontenc}
\usepackage{url}
\usepackage{booktabs}
\usepackage{amsfonts}
\usepackage{nicefrac}
\usepackage{microtype}
\usepackage{lipsum}
\usepackage{graphicx}
\usepackage{natbib}
\usepackage{doi}



\title{Исследование масштабирования размера \\ оптимального батча в больших языковых моделях}

\author{ Бессонов А.С. \\
	МГУ им. М.В. Ломоносова, \\
	кафедра математических методов \\ прогнозирования, 417 группа\\
	\texttt{beccohov.a@yandex.ru} \\
	%% examples of more authors
	\And
        Дьяконов А.Г. \\
	д-р физ.-мат. наук         \\
	Центральный университет Тинькофф \thanks{Научный руководитель} \\
	\texttt{djakonov@mail.ru} \\
	%% \AND
	%% Coauthor \\
	%% Affiliation \\
	%% Address \\
	%% \texttt{email} \\
	%% \And
	%% Coauthor \\
	%% Affiliation \\
	%% Address \\
	%% \texttt{email} \\
	%% \And
	%% Coauthor \\
	%% Affiliation \\
	%% Address \\
	%% \texttt{email} \\
}
\date{}

\renewcommand{\shorttitle}{\textit{arXiv} Template}

%%% Add PDF metadata to help others organize their library
%%% Once the PDF is generated, you can check the metadata with
%%% $ pdfinfo template.pdf
\hypersetup{
pdftitle={A template for the arxiv style},
pdfsubject={q-bio.NC, q-bio.QM},
pdfauthor={David S.~Hippocampus, Elias D.~Striatum},
pdfkeywords={First keyword, Second keyword, More},
}

\begin{document}
\maketitle

\begin{abstract}
В данной статье исследуется связь между оптимальным размером батча при обучении языковых моделей. Нами были проведены эксперименты на широком спектре моделей, варьирующихся от 180M до 4B параметров. Каждая модель обучалась с использованием различных значений размера батча. Количество токенов, применяемых для предварительного обучения модели, зависит только от ее размера и задается величиной $P \times 50$, где $P$ - количество параметров модели. Результаты каждого измерения дали теоретическую оценку оптимального батча для каждой модели, подтверждающую универсальность использованной аппроксимации. Кроме того, была установлена экспериментальная зависимость между размером оптимального теоретического батча и размером модели, а также разработана математическая модель, позволяющая практически оценивать размер батча для больших моделей. Полученные значения приближенно соотносятся с текущими практическими значениями, которые широко принимаются в последнее время.
\end{abstract}


\keywords{Масштабирование моделей \and Языковые модели}
\section{Введение}
\quad Большие языковые модели в последнее десятиление занимают доминирующую роль как в сфере исследований, так и практическом применении. При этом обучение таких моделей требует значительных вычислительных ресурсов, что усложняет эксперименты с подбором гиперпараметров при приобучении. В недавних работах было уделено внимание тому, как масштабируются гиперпараметры подобных моделей при увеличеснии количества параметров. В данной работе исследуется, как должен изменяться размер обучающего батча модели в зависимости от количества параметров в модели.

\quad Недавно стало ясно, что подбор гиперпараметров имеет значительное влияние на качество финальной модели. Например, \cite{hoffmann2022training} указывает на то, что количество обучающих данных должно расти квазилинейно с увеличением размера модели. В \cite{OpenAI2023GPT4TR} также отмечается, что одним из методов для предсказания оптимальных гиперпараметров является построение моделей, способных предсказывать эти гиперпараметры.

\quad Языковые модели начали быстро развиваться с появлением архитектуры трансформера (\cite{vaswani2017attention}). В последствии стало ясно, что главную роль в качестве таких моделей играет их размер (ссылки). Новую эру в развитии языковых  моделей ознаменовало появление архитекруры GPT-2 (\cite{radford2019language}), которая в то время стала SOTA среди языковых моделей. Последующее использование RLHF ознаменовало новую эпоху персональных ассистенов (\cite{ouyang2022training}).


\quad В настоящее время размеры языковых моделей превышают триллионы параметров, что создает множество инженерных трудностей при их обучении. Подбор гиперпараметров для таких моделей практически невозможен. Вместо этого необходимо использовать эвристики и предсказывать оптимальные значения гиперпараметров на основе меньших моделей.

\quad В данной работе исследуется и предлагается эвристика для масштабирования размера обучающего батча. В экспериментах используются оптимальные модели с большим количеством параметров, а размеры моделей варьируются от 180M до 4B параметров. Проведенные эксперименты подтверждают характер масштабирования размера батча и дают оценки оптимальных размеров батчей для очень больших языковых моделей (70B, 140B и более).


\section{Headings: first level}
\label{sec:headings}

\lipsum[4] See Section \ref{sec:headings}.

\subsection{Headings: second level}
\lipsum[5]
\begin{equation}
	\xi _{ij}(t)=P(x_{t}=i,x_{t+1}=j|y,v,w;\theta)= {\frac {\alpha _{i}(t)a^{w_t}_{ij}\beta _{j}(t+1)b^{v_{t+1}}_{j}(y_{t+1})}{\sum _{i=1}^{N} \sum _{j=1}^{N} \alpha _{i}(t)a^{w_t}_{ij}\beta _{j}(t+1)b^{v_{t+1}}_{j}(y_{t+1})}}
\end{equation}

\subsubsection{Headings: third level}
\lipsum[6]

\paragraph{Paragraph}
\lipsum[7]



\section{Examples of citations, figures, tables, references}
\label{sec:others}

\subsection{Citations}
Citations use \verb+natbib+. The documentation may be found at
\begin{center}
	\url{http://mirrors.ctan.org/macros/latex/contrib/natbib/natnotes.pdf}
\end{center}

Here is an example usage of the two main commands (\verb+citet+ and \verb+citep+): Some people thought a thing \citep{kour2014real, hadash2018estimate} but other people thought something else \citep{kour2014fast}. Many people have speculated that if we knew exactly why \citet{kour2014fast} thought this\dots

\subsection{Figures}
\lipsum[10]
See Figure \ref{fig:fig1}. Here is how you add footnotes. \footnote{Sample of the first footnote.}
\lipsum[11]

\begin{figure}
	\centering
	\includegraphics[width=0.5\textwidth]{../figures/log_reg_cs_exp.eps}
	\caption{Sample figure caption.}
	\label{fig:fig1}
\end{figure}

\subsection{Tables}
See awesome Table~\ref{tab:table}.

The documentation for \verb+booktabs+ (`Publication quality tables in LaTeX') is available from:
\begin{center}
	\url{https://www.ctan.org/pkg/booktabs}
\end{center}


\begin{table}
	\caption{Sample table title}
	\centering
	\begin{tabular}{lll}
		\toprule
		\multicolumn{2}{c}{Part}                   \\
		\cmidrule(r){1-2}
		Name     & Description     & Size ($\mu$m) \\
		\midrule
		Dendrite & Input terminal  & $\sim$100     \\
		Axon     & Output terminal & $\sim$10      \\
		Soma     & Cell body       & up to $10^6$  \\
		\bottomrule
	\end{tabular}
	\label{tab:table}
\end{table}

\subsection{Lists}
\begin{itemize}
	\item Lorem ipsum dolor sit amet
	\item consectetur adipiscing elit.
	\item Aliquam dignissim blandit est, in dictum tortor gravida eget. In ac rutrum magna.
\end{itemize}


\bibliographystyle{unsrtnat}
\bibliography{references}

\end{document}
